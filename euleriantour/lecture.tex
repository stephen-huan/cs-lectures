\documentclass[11pt, oneside]{article}
\usepackage{titling, hyperref, geometry, amsmath, amssymb, algorithm, graphicx, textcomp, subcaption}
\usepackage[noend]{algpseudocode}
\usepackage[cache=false]{minted}
\geometry{a4paper}

\hypersetup{
  colorlinks=true,
  urlcolor=cyan
}

\setminted[]{
  linenos=false,
  breaklines=true,
  encoding=utf8,
  fontsize=\normalsize,
  frame=lines,
  framesep=2mm
}

\newcommand{\emphasis}[1]{\textcolor{blue}{\textbf{\textit{#1}}}}

\title{Eulerian Tour}
\author{Stephen Huan \\ Edited by: Udbhav Muthakana}
\date{January 8, 2020}

\begin{document}
\maketitle

\section{Definition}

An \emphasis{Eulerian tour} is path through a graph that uses each edge exactly once.
If it starts and ends at the same node, then it is called an \emphasis{Eulerian circuit}.

\subsection{Identification}
For a graph to have an Eulerian tour, it has to be connected (strongly connected for a directed graph) and either:

\begin{enumerate}
  \item Every node has an even degree (the number of incoming and outgoing edges).
  \item Every node but two has an even degree.
\end{enumerate}

In the second case, the two nodes are the start and end nodes of the circuit.
Detecting strongly connected components can be done in linear time with \href{https://activities.tjhsst.edu/sct/lectures/1920/2019_11_01_Strongly_Connected_Components.pdf}{Kosaraju-Sharir}.

\section{Implementation}

Wikipedia recommends Hierholzer's algorithm but I found it confusing and convoluted.
Just do a postorder iterative DFS on the graph, adding nodes to the tour once their children have been processed.

\section{Applications}

In order to find the lowest common ancestor (LCA) between two nodes in a tree,
one approach is to run a Eulerian tour, doubling up edges between parent and children,
and then do a range minimum query (RMQ) on the resulting tour between the two nodes.

\subsection{Sample Implementation}

\begin{minted}[label=Eulerian tour]{python}
def eulerian_tour(graph):
    # remove nodes with no edges
    graph = {k: v for k, v in graph.items() if len(v) > 0}
    count = sum(len(graph[v]) % 2 for v in graph)
    if count != 0 and count != 2:
        return # no such tour
    start = [v for v in graph if len(graph[v]) % 2 == 1][0] if count == 2 else 0

    stk = [(start, 0)]
    tour = deque([])
    seen = set()
    while len(stk) > 0:
        n, p = stk[-1]
        # done with children
        if len(graph[n]) == p:
            tour.appendleft(n)
            stk.pop()
        # has children left to process
        else:
            child = graph[n][p]
            # edges have a unique id - (vertex, id)
            if child[1] not in seen:
                stk.append((child[0], 0))
                seen.add(child[1])
                stk[-2] = (n, p + 1)
            else:
                stk[-1] = (n, p + 1)
    return list(map(lambda x: x + 1, tour))
\end{minted}

\section{Sample Problems}

\begin{enumerate}
  \item \href{https://train.usaco.org/usacoprob2?a=TpH8RHs6Taa&S=fence}{USACO Training: Riding the Fence}

  Solution: Direct implementation of an Eulerian Tour.
\end{enumerate}

\section{Works Cited}

\begin{enumerate}
  \item \href{https://en.wikipedia.org/wiki/Eulerian_path}{Wikipedia - ``Eulerian path''}
\end{enumerate}

\end{document}
